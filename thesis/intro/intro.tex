%
%%%%%%%%%%%%%%%%%%%%%%%%%%%%%%%%%%%%%%%%%%%%%%%%%%%
%
% E I N L E I T U N G
%
%%%%%%%%%%%%%%%%%%%%%%%%%%%%%%%%%%%%%%%%%%%%%%%%%%%
%
\chapter{Einleitung}
\label{cha:introduction}
%
%
Animation ist die Technik, durch Darstellen einer Sequenz von Einzelbildern, die Illusion von flüssiger Bewegung zu schaffen. Ursprünglich wurden Animationen Bild für Bild erstellt. Mit der Entwicklung der Technik haben sich auch die Methoden zur Animationserstellung verändert. Heutzutage werden Animationen größtenteils mit Hilfe von Computern erstellt. Eine bewährte Technik der Computeranimation ist die Skelettanimation. \\

Im folgenden wird zwischen 3D- und 2D-Animation unterschieden. Als 3D-Animation versteht sich eine Animation, welche mit einem 3D-Modell als Basis erstellt wurde. Bei 2D-Animationen ist dies nicht der Fall. Skelettanimation ist sowohl für 3D-Animation als auch 2D-Animation möglich. Sie wird in den meisten Anwendungsbereichen vorwiegend für 3D-Animationen verwendet z.B. für das Erstellen von 3D-Animationsfilmen. In 2D hat die Skelettanimation einige Nachteile. Da auch 2D-Animationen meist Charaktere in einem 3D-Raum repräsentieren ist zur exakten Abbildung dieser Charaktere auch ein 3D-Modell notwendig. Das ist der Grund dafür, dass 2D Skelettanimationen oft etwas unnatürlich wirken. \\

Dennoch ist die 2D-Skelettanimation in der Entwicklung von 2D-Videospielen weit verbreitet, da sie hier im Vergleich zur Bild-zu-Bild-Animation einige Vorteile mit sich bringt. Da die Skelettanimation weniger arbeitsintensiv als die Bild-zu-Bild-Animation und ermöglicht effiziente Abänderungen der Animation im Nachhinein und sogar zur Laufzeit. Des Weiteren ermöglicht die Skelettanimation eine größere Interaktion des animierten Charakters mit der Spielwelt. Die vorliegende Arbeit ist eine Untersuchung verbreiteter Techniken zur 2D-Skelettanimation und deren Anwendung in Videospielen. \\
