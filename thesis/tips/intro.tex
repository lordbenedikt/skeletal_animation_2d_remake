%
%%%%%%%%%%%%%%%%%%%%%%%%%%%%%%%%%%%%%%%%%%%%%%%%%%%
%
% E I N L E I T U N G
%
%%%%%%%%%%%%%%%%%%%%%%%%%%%%%%%%%%%%%%%%%%%%%%%%%%%
%
\chapter{Einleitung}
\label{cha:introduction}
%
%
LaTeX Elemente: \\

\noindent \textbf{\textcolor{darkred}{Hinweis}}: Dies ist keine Anleitung... \\
\textit{memoir} \\
\texttt{ab"-schluss"-ar"-beit"-.tex} \\
(siehe auch Abschnitt \ref{sec:abbildungen}). \\

Animation ist die Technik durch Darstellen einer Sequenz von Einzelbildern die Illusion von flüssiger Bewegung zu schaffen. Ursprünglich wurden Animationen Bild für Bild erstellt. Mit der Entwicklung der Technik haben sich auch die Methoden zur Animationserstellung verändert. Heutzutage werden Animationen größtenteils mit Hilfe von Computern erstellt. Eine bewährte Technik der Computeranimation ist die Skelettanimation. \\

Im folgenden wird zwischen 3D- und 2D-Animation unterschieden. Als 3D-Animation versteht sich eine Animation, welche mit einem 3D-Modell als Basis erstellt wurde. Bei 2D-Animationen ist dies nicht der Fall. Skelettanimation ist sowohl für 3D-Animation als auch 2D-Animation möglich. Sie wird in den meisten Anwendungsbereichen vorwiegend für 3D-Animationen verwendet z.B. für das Erstellen von 3D-Animationsfilmen. In 2D hat die Skelettanimation einige Nachteile. Da auch 2D-Animationen meist Charaktere in einem 3D-Raum repräsentieren ist zur exakten Abbildung dieser Charaktere auch ein 3D-Modell notwendig. Das ist der Grund dafür, dass 2D Skelettanimationen oft etwas unnatürlich wirken. \\

Dennoch ist die 2D-Skelettanimation in der Entwicklung von 2D-Videospielen weit verbreitet, da sie hier im Vergleich zur Bild-zu-Bild-Animation einige Vorteile mit sich bringt. Da die Skelettanimation weniger arbeitsintensiv als die Bild-zu-Bild-Animation und ermöglicht effiziente Abänderungen der Animation im Nachhinein und sogar zur Laufzeit. Des Weiteren ermöglicht die Skelettanimation eine größere Interaktion des animierten Charakters mit der Spielwelt. Die vorliegende Arbeit ist eine Untersuchung verbreiteter Techniken zur 2D-Skelettanimation und deren Anwendung in Videospielen. \\
%
%
%%%%%%%%%%%%%%%%%%%%%%%%%%%%%%%%%%%%%%%%%%%%%%%%%%%
%
% A U F B A U 
%
%%%%%%%%%%%%%%%%%%%%%%%%%%%%%%%%%%%%%%%%%%%%%%%%%%%
%
\section{Aufbau des Dokuments}
\label{sec:aufbau}
%

%
%
%%%%%%%%%%%%%%%%%%%%%%%%%%%%%%%%%%%%%%%%%%%%%%%%%%%
%
% Abbildungen
%
%%%%%%%%%%%%%%%%%%%%%%%%%%%%%%%%%%%%%%%%%%%%%%%%%%%
%
\section{Abbildungen}
\label{sec:abbildungen}
%
Bilder sollten im \texttt{.eps}-Format (Encapsulated PostScript) eingebunden werden. Wandeln Sie daher eine Grafik, die als \texttt{.jpg}, \texttt{.png} oder in einem anderen Dateiformat vorliegt, vor der Verwendung in Ihrer Bachelorarbeit entsprechend um. Alle gängigen Bildbearbeitungsprogramme wie beispielsweise Adobe Photoshop \footnote{http://www.adobe.com/de/products/photoshop/family/} oder Gimp \footnote{http://www.gimp.org/} sind in der Lage, Bilder ins \texttt{EPS}-Format zu exportieren.
%


%
Wenn Sie auf Abbildungen im Text verweisen wollen, verwenden Sie die \texttt{ref}-Marke: Die \textit{Kleinsche Flasche} \cite{schrijver89} (siehe Abbildung \ref{fig:klein_bottle}) ist ein geometrisches Objekt, das nur eine einzige Seite besitzt. Man kann also nicht zwischen \textit{innen} und \textit{außen} unterscheiden. Beschriften Sie die Abbildungen ausführlich. Verwenden Sie die beiden Textoptionen in der \texttt{caption}-Marke, um auch eine kurze Beschreibung für das Abbildungsverzeichnis zu erzeugen. \\
%
Vergessen Sie bitte nicht, die Eigentümer eines Bildes zu nennen, wenn Sie ein Bild nicht selbst erstellt haben. \\
%
%
%%%%%%%%%%%%%%%%%%%%%%%%%%%%%%%%%%%%%%%%%%%%%%%%%%%
%
%  L I T E R A T U R V E R W E I S E
%
%%%%%%%%%%%%%%%%%%%%%%%%%%%%%%%%%%%%%%%%%%%%%%%%%%%
%
\section{Literaturverweise}
\label{sec:verweise}
%
Verweise auf Literatur, die Sie in der Ihrer Bachelorarbeit verwenden, sollten in der Datei \texttt{abschlussarbeit.bib} eingetragen werden. In dieser Datei können durchaus mehr Einträge enthalten sein, als Sie in Ihrer Arbeit als Referenzen verwenden. LaTeX wird nur auch tatsächlich referenzierte Literatur in das Verzeichnis am Ende der Arbeit aufnehmen. Die freie Software LEd (siehe Kapitel \ref{cha:umgebung}) ist in der Lage, Vorlagen für die unterschiedlichsten Arten von Literatur aufzunehmen, wie beispielsweise Bücher, Artikel in Zeitschriften oder wissenschaftliche Veröffentlichungen in den \textit{Proceedings} von Konferenzen. Denken Sie bitte bei Ihrer Recherchearbeit daran, dass Sie von Rechnern der Hochschule aus freien Zugang zu einer Vielzahl von Online-Bibliotheken wie IEEE\footnote{http://ieeexplore.ieee.org/} und ACM\footnote{http://portal.acm.org/} haben.
%
%
%%%%%%%%%%%%%%%%%%%%%%%%%%%%%%%%%%%%%%%%%%%%%%%%%%%
%
%   F O R M E L N   U N D   T A B E L L E N
%
%%%%%%%%%%%%%%%%%%%%%%%%%%%%%%%%%%%%%%%%%%%%%%%%%%%
%
\section{Verwendung von Formeln und Tabellen}
\label{sec:formeln}
%
Während Tabellen leider schnell unübersichtlich werden können, spielt LaTeX bei Formeln sein volles Potential aus. Kein anderes Textsatzsystem erzeugt so schnell und einfach schöne Formeln. Im folgenden Abschnitt werden einige Textpassagen vorgestellt, die typische Formeln verwenden. Ein Blick an die entsprechenden Zeilen in der LaTeX-Datei zeigt Ihnen, wie diese Formeln entstehen.\\
\\
Formeln können direkt in den Text integriert werden: $x_i=x_{x-1} + x_{i-2}$ ist beispielsweise die Rechenvorschrift für die Fibonacci-Folge \cite{conway96}. \\
\\
``...Das Kameramodell basiert auf dem Prinzip der Lochkamera und wird für eine präzise Triangulierung um intrinsische Parameter wie etwa der Linsenverzeichnung erweitert: \\

\noindent \textbf{\textcolor{darkred}{Hinweis}}: Das $\Real$-Zeichen und die dunkelrote Textfarbe sind im Hauptdokument \texttt{ab"-schluss"-ar"-beit"-.tex} definiert.
\begin{tabbing}
	Platzhalter links \quad \= Platzhalter Mitte \quad \= Platzhalter rechts \kill
	$T \in \Real^3$  \> der Brennpunkt und Ursprung des Kamerakoordinatensystems \\
	\> im Weltkoordinatensystem und                             \\
	$R \in SO_3$     \> die Rotation der Kamera im Weltkoordinatensystem
\end{tabbing}
Als intrinsische Parameter werden
\begin{tabbing}
	Platzhalter links \quad \= Platzhalter Mitte \quad \= Platzhalter rechts \kill
	$f \in \Real$             \> die Brennweite (fokale Länge) der Kamera,                                          \\
	$P=(u_0,v_0) \in \Real^2$ \> der Hauptpunkt des Bildes (engl.: \textit{principle point}), also der Schnittpunkt \\
	\> der optischen Achse mit der dazu orthogonal stehenden Bildebene,                   \\
	$r_u,r_v \in \Real$       \> Skalierungsfaktoren in x- und y-Richtung auf der Bildebene,                        \\
	$s \in \Real$             \> ein Verzerrungsfaktor (engl.: \textit{skew}) der Kameralinse und                   \\
	$\kappa \in \Real$        \> die radiale Verzeichnung der Kameralinse
\end{tabbing}
gewählt.\\
%
Mit Hilfe der Strahlensätze ist die Projektion durch $(\nicefrac{-f \cdot x_c}{z_c}, \nicefrac{-f \cdot y_c}{z_c})^T$ zu berechnen. Die restlichen intrinsischen Kameraparameter lassen sich in der Kamerakalibrierungsmatrix $K$ mit
\begin{equation}
	\label{eqn:K}
	\left(\begin{array}{ccc}r_u&s&u_0\\0&r_v&v_0\\0&0&1 \end{array}\right) \cdot 
	\left(\begin{array}{c}\nicefrac{-f \cdot x_c}{z_c}\\\nicefrac{-f \cdot y_c}{z_c}\\1\end{array}\right) = 
	\underbrace{\left(\begin{array}{ccc}-f \cdot r_u&-f \cdot s&u_0\\0&-f \cdot r_v&v_0\\0&0&1 \end{array}\right)}_{=K} \cdot 
	\left(\begin{array}{c}\nicefrac{x_c}{z_c}\\\nicefrac{y_c}{z_c}\\1\end{array}\right)
\end{equation}
zusammenfassen. Diese Matrix wird so genannt, da alle intrinsischen Konstanten der Kalibrierung in einer einzigen Matrix enthalten sind. \\
%
Die Summe der $m$ Abstände zwischen Originalpunkt und projiziertem Punkt in der Bildebene der zweiten Kamera wird mit
\begin{equation}
	\sum_{i=1..m}r_i = \sum_{i=1..m} \Vert\tilde U^2_i - U^2_i \Vert
\end{equation}
minimiert.''\\
%

Nun ein Beispiel, bei dem zwei Formeln in einer Zeile stehen, das verbindende \textit{UND} aber nicht als Formeltext, sondern als normaler Flusstext erscheint: ``...Als erster Schritt der Rekonstruktion werden beide 2D-Punkte um einen Tiefenwert erweitert, um 3D-Koordinaten zu erhalten:''
%
\begin{equation}
	\bar{U}^1=\left(\begin{array}{c}U^1\\-f_1\end{array}\right) \in \Real^3 \quad \mbox{und} \quad
	\bar{U}^2=\left(\begin{array}{c}U^2\\-f_2\end{array}\right) \in \Real^3
\end{equation}
%

Am Ende dieses Abschnitts noch ein Beispiel für eine Tabelle:\\
``Die folgende Tabelle \ref{tab:vergleich} stellt die vier Verfahren in einem direkten Vergleich in Bezug auf die drei Parameter gegenüber:''
%
%
%  T A B L E   V E R G L E I C H
%
\begin{table}[H]
	\centering{
		\caption{Vergleich verschiedener Algorithmen.}
		\label{tab:vergleich}
		\begin{tabular}{p{2.5cm}p{2.5cm}p{2.5cm}p{2.5cm}p{2.5cm}}
			\hline \addlinespace
			& \textbf{Verfahren 1} & \textbf{Verfahren 2} & \textbf{Verfahren 3} & \textbf{Verfahren 4} \\
			\addlinespace \hline \addlinespace
			\textbf{Lexikoneintrag} & nein & ja & ja & nein \\
			\addlinespace \hline \addlinespace
			\textbf{Trainings\-aufwand} & keiner & keiner & 2 min bei \newline Optimierung\textsuperscript{1} & keiner \\
			\addlinespace \hline \addlinespace
			\textbf{Ergebnis} & 22 sec & 32 sec & 30 sec & 13 sec \\
			\addlinespace \hline
	\end{tabular}}
\end{table}
%\vspace{1mm}
\noindent \textsuperscript{1} Anmerkungen zu Tabelleneinträgen können am Ende der Tabelle erklärt werden. \vspace{2mm} \\
%
Lorem ipsum dolor sit amet, consectetur adipisici elit, sed eiusmod tempor incidunt ut labore et dolore magna aliqua. Ut enim ad minim veniam, quis nostrud exercitation ullamco laboris nisi ut aliquid ex ea commodi consequat. Quis aute iure reprehenderit in voluptate velit esse cillum dolore eu fugiat nulla pariatur. Excepteur sint obcaecat cupiditat non proident, sunt in culpa qui officia deserunt mollit anim id est laborum. \\
%
%
%%%%%%%%%%%%%%%%%%%%%%%%%%%%%%%%%%%%%%%%%%%%%%%%%%%
%
%   T I P P S
%
%%%%%%%%%%%%%%%%%%%%%%%%%%%%%%%%%%%%%%%%%%%%%%%%%%%
%
\section{Tipps}
\label{sec:tipps}
%
In dieser Vorlage ist die automatische Silbentrennung bereits aktiviert. Durch andere Befehle kann es passieren, dass hier eingebettete Wörter aber nicht automatisch getrennt werden. Hier ein kleines Beispiel:\\
\\
Wenn man einen längeren Satz oder Absatz schreibt, der über mehrere Zeilen sich dann verteilt, kann es natürlich vorkommen, dass ein in TrueType geschriebenes Wort \texttt{abschlussarbeit.tex} (im Source Code \texttt{\textbackslash texttt\{abschlussarbeit.tex\}}) über den Zeilenrand hinaus ragt und dann nicht nur unschön aussieht, sondern eventuell sogar Teile des Wortes abgeschnitten werden. Dies passiert insbesondere oft bei langen Weblinks.\\
\\
Hier hilft es, alle im Wort möglichen Trennungen mit \dq- anzugeben, also im Quelltext \texttt{\textbackslash texttt\{ab\dq-schluss\dq-ar\dq-beit\dq-.tex\}}. Das führt dann zu dem richtigen Ergebnis:\\
\\
\textcolor{darkblue}{Wenn man einen längeren Satz oder Absatz schreibt, der über mehrere Zeilen sich dann verteilt, kann es nicht vorkommen, dass ein in TrueType geschriebenes Wort \texttt{ab"-schluss"-ar"-beit"-.tex} über den Rand hinausgeht, wenn man alle möglichen Trennpositionen im Quelltext vorgibt.}\\
\\
